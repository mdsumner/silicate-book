\documentclass[]{book}
\usepackage{lmodern}
\usepackage{amssymb,amsmath}
\usepackage{ifxetex,ifluatex}
\usepackage{fixltx2e} % provides \textsubscript
\ifnum 0\ifxetex 1\fi\ifluatex 1\fi=0 % if pdftex
  \usepackage[T1]{fontenc}
  \usepackage[utf8]{inputenc}
\else % if luatex or xelatex
  \ifxetex
    \usepackage{mathspec}
  \else
    \usepackage{fontspec}
  \fi
  \defaultfontfeatures{Ligatures=TeX,Scale=MatchLowercase}
\fi
% use upquote if available, for straight quotes in verbatim environments
\IfFileExists{upquote.sty}{\usepackage{upquote}}{}
% use microtype if available
\IfFileExists{microtype.sty}{%
\usepackage{microtype}
\UseMicrotypeSet[protrusion]{basicmath} % disable protrusion for tt fonts
}{}
\usepackage[margin=1in]{geometry}
\usepackage{hyperref}
\hypersetup{unicode=true,
            pdftitle={A Minimal Book Example},
            pdfauthor={Yihui Xie},
            pdfborder={0 0 0},
            breaklinks=true}
\urlstyle{same}  % don't use monospace font for urls
\usepackage{natbib}
\bibliographystyle{apalike}
\usepackage{color}
\usepackage{fancyvrb}
\newcommand{\VerbBar}{|}
\newcommand{\VERB}{\Verb[commandchars=\\\{\}]}
\DefineVerbatimEnvironment{Highlighting}{Verbatim}{commandchars=\\\{\}}
% Add ',fontsize=\small' for more characters per line
\usepackage{framed}
\definecolor{shadecolor}{RGB}{248,248,248}
\newenvironment{Shaded}{\begin{snugshade}}{\end{snugshade}}
\newcommand{\KeywordTok}[1]{\textcolor[rgb]{0.13,0.29,0.53}{\textbf{#1}}}
\newcommand{\DataTypeTok}[1]{\textcolor[rgb]{0.13,0.29,0.53}{#1}}
\newcommand{\DecValTok}[1]{\textcolor[rgb]{0.00,0.00,0.81}{#1}}
\newcommand{\BaseNTok}[1]{\textcolor[rgb]{0.00,0.00,0.81}{#1}}
\newcommand{\FloatTok}[1]{\textcolor[rgb]{0.00,0.00,0.81}{#1}}
\newcommand{\ConstantTok}[1]{\textcolor[rgb]{0.00,0.00,0.00}{#1}}
\newcommand{\CharTok}[1]{\textcolor[rgb]{0.31,0.60,0.02}{#1}}
\newcommand{\SpecialCharTok}[1]{\textcolor[rgb]{0.00,0.00,0.00}{#1}}
\newcommand{\StringTok}[1]{\textcolor[rgb]{0.31,0.60,0.02}{#1}}
\newcommand{\VerbatimStringTok}[1]{\textcolor[rgb]{0.31,0.60,0.02}{#1}}
\newcommand{\SpecialStringTok}[1]{\textcolor[rgb]{0.31,0.60,0.02}{#1}}
\newcommand{\ImportTok}[1]{#1}
\newcommand{\CommentTok}[1]{\textcolor[rgb]{0.56,0.35,0.01}{\textit{#1}}}
\newcommand{\DocumentationTok}[1]{\textcolor[rgb]{0.56,0.35,0.01}{\textbf{\textit{#1}}}}
\newcommand{\AnnotationTok}[1]{\textcolor[rgb]{0.56,0.35,0.01}{\textbf{\textit{#1}}}}
\newcommand{\CommentVarTok}[1]{\textcolor[rgb]{0.56,0.35,0.01}{\textbf{\textit{#1}}}}
\newcommand{\OtherTok}[1]{\textcolor[rgb]{0.56,0.35,0.01}{#1}}
\newcommand{\FunctionTok}[1]{\textcolor[rgb]{0.00,0.00,0.00}{#1}}
\newcommand{\VariableTok}[1]{\textcolor[rgb]{0.00,0.00,0.00}{#1}}
\newcommand{\ControlFlowTok}[1]{\textcolor[rgb]{0.13,0.29,0.53}{\textbf{#1}}}
\newcommand{\OperatorTok}[1]{\textcolor[rgb]{0.81,0.36,0.00}{\textbf{#1}}}
\newcommand{\BuiltInTok}[1]{#1}
\newcommand{\ExtensionTok}[1]{#1}
\newcommand{\PreprocessorTok}[1]{\textcolor[rgb]{0.56,0.35,0.01}{\textit{#1}}}
\newcommand{\AttributeTok}[1]{\textcolor[rgb]{0.77,0.63,0.00}{#1}}
\newcommand{\RegionMarkerTok}[1]{#1}
\newcommand{\InformationTok}[1]{\textcolor[rgb]{0.56,0.35,0.01}{\textbf{\textit{#1}}}}
\newcommand{\WarningTok}[1]{\textcolor[rgb]{0.56,0.35,0.01}{\textbf{\textit{#1}}}}
\newcommand{\AlertTok}[1]{\textcolor[rgb]{0.94,0.16,0.16}{#1}}
\newcommand{\ErrorTok}[1]{\textcolor[rgb]{0.64,0.00,0.00}{\textbf{#1}}}
\newcommand{\NormalTok}[1]{#1}
\usepackage{longtable,booktabs}
\usepackage{graphicx,grffile}
\makeatletter
\def\maxwidth{\ifdim\Gin@nat@width>\linewidth\linewidth\else\Gin@nat@width\fi}
\def\maxheight{\ifdim\Gin@nat@height>\textheight\textheight\else\Gin@nat@height\fi}
\makeatother
% Scale images if necessary, so that they will not overflow the page
% margins by default, and it is still possible to overwrite the defaults
% using explicit options in \includegraphics[width, height, ...]{}
\setkeys{Gin}{width=\maxwidth,height=\maxheight,keepaspectratio}
\IfFileExists{parskip.sty}{%
\usepackage{parskip}
}{% else
\setlength{\parindent}{0pt}
\setlength{\parskip}{6pt plus 2pt minus 1pt}
}
\setlength{\emergencystretch}{3em}  % prevent overfull lines
\providecommand{\tightlist}{%
  \setlength{\itemsep}{0pt}\setlength{\parskip}{0pt}}
\setcounter{secnumdepth}{5}
% Redefines (sub)paragraphs to behave more like sections
\ifx\paragraph\undefined\else
\let\oldparagraph\paragraph
\renewcommand{\paragraph}[1]{\oldparagraph{#1}\mbox{}}
\fi
\ifx\subparagraph\undefined\else
\let\oldsubparagraph\subparagraph
\renewcommand{\subparagraph}[1]{\oldsubparagraph{#1}\mbox{}}
\fi

%%% Use protect on footnotes to avoid problems with footnotes in titles
\let\rmarkdownfootnote\footnote%
\def\footnote{\protect\rmarkdownfootnote}

%%% Change title format to be more compact
\usepackage{titling}

% Create subtitle command for use in maketitle
\newcommand{\subtitle}[1]{
  \posttitle{
    \begin{center}\large#1\end{center}
    }
}

\setlength{\droptitle}{-2em}
  \title{A Minimal Book Example}
  \pretitle{\vspace{\droptitle}\centering\huge}
  \posttitle{\par}
  \author{Yihui Xie}
  \preauthor{\centering\large\emph}
  \postauthor{\par}
  \predate{\centering\large\emph}
  \postdate{\par}
  \date{2017-09-08}

\usepackage{booktabs}
\usepackage{amsthm}
\makeatletter
\def\thm@space@setup{%
  \thm@preskip=8pt plus 2pt minus 4pt
  \thm@postskip=\thm@preskip
}
\makeatother

\usepackage{amsthm}
\newtheorem{theorem}{Theorem}[chapter]
\newtheorem{lemma}{Lemma}[chapter]
\theoremstyle{definition}
\newtheorem{definition}{Definition}[chapter]
\newtheorem{corollary}{Corollary}[chapter]
\newtheorem{proposition}{Proposition}[chapter]
\theoremstyle{definition}
\newtheorem{example}{Example}[chapter]
\theoremstyle{definition}
\newtheorem{exercise}{Exercise}[chapter]
\theoremstyle{remark}
\newtheorem*{remark}{Remark}
\newtheorem*{solution}{Solution}
\begin{document}
\maketitle

{
\setcounter{tocdepth}{1}
\tableofcontents
}
\chapter{Prerequisites}\label{prerequisites}

This is a \emph{sample} book written in \textbf{Markdown}. You can use
anything that Pandoc's Markdown supports, e.g., a math equation
\(a^2 + b^2 = c^2\).

For now, you have to install the development versions of
\textbf{bookdown} from Github:

\begin{Shaded}
\begin{Highlighting}[]
\NormalTok{devtools}\OperatorTok{::}\KeywordTok{install_github}\NormalTok{(}\StringTok{"rstudio/bookdown"}\NormalTok{)}
\end{Highlighting}
\end{Shaded}

Remember each Rmd file contains one and only one chapter, and a chapter
is defined by the first-level heading \texttt{\#}.

To compile this example to PDF, you need to install XeLaTeX.

\chapter{Introduction}\label{intro}

You can label chapter and section titles using \texttt{\{\#label\}}
after them, e.g., we can reference Chapter \ref{intro}. If you do not
manually label them, there will be automatic labels anyway, e.g.,
Chapter \ref{methods}.

Figures and tables with captions will be placed in \texttt{figure} and
\texttt{table} environments, respectively.

\begin{Shaded}
\begin{Highlighting}[]
\KeywordTok{par}\NormalTok{(}\DataTypeTok{mar =} \KeywordTok{c}\NormalTok{(}\DecValTok{4}\NormalTok{, }\DecValTok{4}\NormalTok{, .}\DecValTok{1}\NormalTok{, .}\DecValTok{1}\NormalTok{))}
\KeywordTok{plot}\NormalTok{(pressure, }\DataTypeTok{type =} \StringTok{'b'}\NormalTok{, }\DataTypeTok{pch =} \DecValTok{19}\NormalTok{)}
\end{Highlighting}
\end{Shaded}

\begin{figure}

{\centering \includegraphics[width=0.8\linewidth]{bookdown-demo_files/figure-latex/nice-fig-1} 

}

\caption{Here is a nice figure!}\label{fig:nice-fig}
\end{figure}

Reference a figure by its code chunk label with the \texttt{fig:}
prefix, e.g., see Figure \ref{fig:nice-fig}. Similarly, you can
reference tables generated from \texttt{knitr::kable()}, e.g., see Table
\ref{tab:nice-tab}.

\begin{Shaded}
\begin{Highlighting}[]
\NormalTok{knitr}\OperatorTok{::}\KeywordTok{kable}\NormalTok{(}
  \KeywordTok{head}\NormalTok{(iris, }\DecValTok{20}\NormalTok{), }\DataTypeTok{caption =} \StringTok{'Here is a nice table!'}\NormalTok{,}
  \DataTypeTok{booktabs =} \OtherTok{TRUE}
\NormalTok{)}
\end{Highlighting}
\end{Shaded}

\begin{table}

\caption{\label{tab:nice-tab}Here is a nice table!}
\centering
\begin{tabular}[t]{rrrrl}
\toprule
Sepal.Length & Sepal.Width & Petal.Length & Petal.Width & Species\\
\midrule
5.1 & 3.5 & 1.4 & 0.2 & setosa\\
4.9 & 3.0 & 1.4 & 0.2 & setosa\\
4.7 & 3.2 & 1.3 & 0.2 & setosa\\
4.6 & 3.1 & 1.5 & 0.2 & setosa\\
5.0 & 3.6 & 1.4 & 0.2 & setosa\\
\addlinespace
5.4 & 3.9 & 1.7 & 0.4 & setosa\\
4.6 & 3.4 & 1.4 & 0.3 & setosa\\
5.0 & 3.4 & 1.5 & 0.2 & setosa\\
4.4 & 2.9 & 1.4 & 0.2 & setosa\\
4.9 & 3.1 & 1.5 & 0.1 & setosa\\
\addlinespace
5.4 & 3.7 & 1.5 & 0.2 & setosa\\
4.8 & 3.4 & 1.6 & 0.2 & setosa\\
4.8 & 3.0 & 1.4 & 0.1 & setosa\\
4.3 & 3.0 & 1.1 & 0.1 & setosa\\
5.8 & 4.0 & 1.2 & 0.2 & setosa\\
\addlinespace
5.7 & 4.4 & 1.5 & 0.4 & setosa\\
5.4 & 3.9 & 1.3 & 0.4 & setosa\\
5.1 & 3.5 & 1.4 & 0.3 & setosa\\
5.7 & 3.8 & 1.7 & 0.3 & setosa\\
5.1 & 3.8 & 1.5 & 0.3 & setosa\\
\bottomrule
\end{tabular}
\end{table}

You can write citations, too. For example, we are using the
\textbf{bookdown} package \citep{R-bookdown} in this sample book, which
was built on top of R Markdown and \textbf{knitr} \citep{xie2015}.

\section{Purpose}\label{purpose}

We aim to create a classification of spatial and other hierarchical data
in R with tools for more general representations of spatial primitives
and the intermediate forms required for translation and analytical
tasks. The key is to provide a relational model of indexed primitives
and component elements, as a bridge to the traditionally
\emph{structural}, or \emph{array/matrix} indexing and storage used in
computer graphics and gaming.

A \textbf{path} can be treated as a first-class type and and stored as
such within a relational model, along with the other entities
\textbf{objects} (``features'') and \textbf{vertices}. with this
approach we gain two advantages, we can \emph{normalize} the relations
(detect and remove redundancy) and also store any additional data about
the entitities in the model.

(There is an interplay between ``able to store extra information'' and
``able to normalize'', since extra data may introduce further
redundancy, but we defer this issue for now since full normalization is
not our primary task).

This package provides two main schemes, the \texttt{PATH} and the
\texttt{PRIMITIVE} models. The two models may be mutually exclusive, or
they can co-exist. The PATH model can always be derived from the
PRIMITIVE model and vice versa but the PRIMITIVE model has extra
capacities that PATH cannot provide. (The current design is a
distillation and improvement on previous implementations, see
\texttt{mdsumner/gris} and \texttt{r-gris/rangl} for these earlier
attempts).

\chapter{Why?}\label{why}

Geographic Information System (GIS) tools provide data structures
optimized for a relatively narrow class of workflows that leverage a
combination of \emph{spatial}, graphics, drawing-design, imagery,
geodetic and database techniques. When modern GIS was born in the 1990s
it adopted a set of compromises that divorced it from its roots in graph
theory (arc-node topology) to provide the best performance for what were
the most complicated sets of cartographic and land-management system
data at the time.

The huge success of ArcView and the shapefile brought this arcane domain
into common usage and helped establish our modern view of what
``geo-spatial data'' is. The creation of the ``simple features
standard''" in the early 2000s formalized this modern view and provided
a basis to avoid some of the inconsistencies and incompleteness that are
present in the shapefile specification.

(For some background, see here
\url{http://www.esri.com/news/arcuser/0401/topo.html})

Spatial, graphics, drawing-design, imagery, geodetic and database
techniques are broader than any GIS, are used in combination in many
fields, but no other field combines them in the way that GIS tools do.
GIS does however impose a certain view point, a lens through which each
of those very general fields is seen via the perspective of the
optimizations, the careful constraints and compromises that were
formalized in the early days.

This lens is seen side-on when 1) bringing graphics data (images,
drawings) into a GIS where a localization metadata context is assumed 2)
attempting to visualize geo-spatial raster data with graphics tools 3)
creating lines to represent the path of sensor platforms that record
many variables like temperature, salinity, radiative flux as well as
location in time and space.

The word ``spatial'' has a rather general meaning, and while GIS idioms
sometimes extend into the Z dimension time is usually treated in a
special way. Where GIS really starts to show its limits is in the
boundary between discrete and continuous measures and entities. We
prefer to default to the most general meaning of spatial, work with
tools that allow flexibility despite the (rather arbitrary) choice of
topological and geometric structures and dimensions that a given model
needs. When the particular optimizations and clever constraints of the
simple features and GIS world are required and/or valuable then we use
those, but prefer not to see that 1) this model must fit into this GIS
view 2) GIS has no place in this model. For us the boundaries are not so
sharp and there's valuable cross-over in many fields.

The particular GIS-like limitations that we seek to overcome are as
follows.

\begin{itemize}
\tightlist
\item
  flexibility in the number and type/s of attribute stored as
  ``coordinates'', x, y, lon, lat, z, time, temperature, etc.
\item
  ability to store attributes on parts i.e.~the state is the object, the
  county is the part
\item
  shared vertices
\item
  the ability to leverage topology engines like D3 to dynamically
  segmentize a piecewise graph using geodetic curvature
\item
  the ability to extend the hierarchical view in GIS to 3D, 4D spatial,
  graphical, network and general modelling domains
\item
  clarity on the distinction between topology and geometry
\item
  clarity on the distinction between vector and raster data, without
  having an arbitrary boundary between them
\item
  multiple models of raster \texttt{georeferencing} from basic
  offset/scale, general affine transform, full curvilinear and partial
  curvilinear with affine and rectilinear optimizations where applicable
\item
  ability to store points, lines and areas together, with shared
  topology as appropriate
\item
  a flexible and powerful basis for conversion between formats both in
  the GIS idioms and outside them
\item
  flexibility, ease-of-use, composability, modularity, tidy-ness
\item
  integration with specialist computational engines, database systems,
  geometric algorithms, drawing tools and other systems
\item
  interactivity, integration with D3, shiny, ggplot2, ggvis, leaflet
\item
  scaleability, the ability to leverage back-end databases, specialist
  parallelism engines, remote compute and otherwise distributed compute
  systems
\end{itemize}

Flexibility in attributes generally is the key to breaking out of
traditional GIS constraints that don't allow clear continuous / discrete
distinctions, or time-varying objects/events, 3D/4D geometry, or clarity
on topology versus geometry. When everything is tables this becomes
natural, and we can build structures like link-relations between tables
that transfer data only when required.

The ability many GIS tools from R in a consistent way is long-term goal,
and this will be best done via dplyr ``back-ending'' or a model very
like it.

\chapter{Approach}\label{approach}

We can't possibly provide all the aspirations detailed above, but we
hope to

\begin{itemize}
\tightlist
\item
  demonstrate the clear need, interest and opportunities that currently
  exist for fostering their development
\item
  illustrate links between existing systems that from a slightly
  different perspective become achievable goals rather than
  insurmountable challenges
\item
  provide a platform for generalizing some of the currently fragmented
  translations that occur across the R community between commonly used
  tools that aren't formally speaking to each other.
\item
  provide tools that we build along the way
\end{itemize}

This package is intended to provide support to the \texttt{common\ form}
approach described here. The package is not fully functional yet, but
see these projects that are informed by this approach.

\begin{itemize}
\tightlist
\item
  \textbf{rbgm} -
  \href{https://github.com/AustralianAntarcticDivision/rbgm}{Atlantis
  Box Geometry Model}, a ``doubly-connected edge-list'' form of linked
  faces and boxes in a spatially-explicit 3D ecosystem model
\item
  \textbf{rangl} - \href{https://github.com/r-gris/rangl}{Primitives for
  Spatial data}, a generalization of GIS forms with simple 3D plotting
\item
  \textbf{spbabel} -
  \href{https://github.com/mdsumner/spbabel}{Translators for R Spatial},
  tools to convert from and to spatial forms, provides the general
  decomposition framework for paths, used by \texttt{rangl}
\item
  \textbf{sfct} - \href{https://github.com/r-gris/sfct}{Constrained
  Triangulation for Simple Features} tools to decompose
  \texttt{simple\ features} into (non-mesh-indexed) primitives.
\end{itemize}

\chapter{Applications}\label{applications}

Some \emph{significant} applications are demonstrated in this chapter.

\section{Example one}\label{example-one}

\section{Example two}\label{example-two}

\bibliography{packages.bib,book.bib}


\end{document}
